\section{Portal Web científico}


Un portal web es una aplicación para el acceso y uso a través de un navegador que reúne distintas aplicaciones, recursos y usuarios los cuales se encuentran distribuidos remotamente. El término fue introducido gracias a la NCSA (National Computational Science Alliance) como parte de un proyecto diseñado para facilitar a biólogos el acceso a herramientas avanzadas y bases de datos que pudieran ser compartidas con la comunidad a través de la web. Dichos portales tiene como finalidad facilitar el acceso del usuario pero también incrementar la interoperabilidad entre los sistemas grid que representan, entre ellos, portales científicos. \\

Los portales ofrecen como servicio la seguridad, el manejo de la información, la publicación trabajos, interfaces gráficas para las aplicaciones, herramientas colaborativas, administración de workflow y visualización de datos\cite{Thomas:2005}. \\

Existen dos tipos de portales, aquellos que ofrecen servicios específicos a la comunidad de científica y aquellos que proveen mecanismos más genéricos para acceder a los recursos grid. Existen tecnologías necesarias para la producción e implementación de portales que se conocen como meta portales, o portales usados en la creación de nuevos portales\cite{Wiley:2006} \cite{Thomas:2006}. \\

La literatura sobre los portales científicos se ha limitado a una descripción del funcionamiento de los mismos más que una descripción de los componentes, sin embargo, los pocos artículos que existen, intentan realizar un esbozo sobre las características que tiene en común así como una clasificación general sobre los servicios que ofrecen.\\

Existen numerosos portales para la computación científica, estos ofrecen interfaces web accesibles a través de cualquier navegador o a través de programas que los mismos desarrolladores crean.\\

En latinoamerica se destaca “GISELA-GRID”, un esfuerzo mancomunado de la Universidad Autónoma de México, el Centro Universitario de Desarrollo Intelectual (CUDI), y la Universidad de los Andes en Venezuela, el cual provee un “conjunto de herramientas, datos y aplicaciones de computación avanzada disponible a las comunidades de investigación de América Latina.” Atendiendo así, las necesidades de cómputo de distintas comunidades académicas a lo largo del continente\cite{Gisela}.\\

Otra aproximación que se usa en la comunidad científica para consolidar recursos informáticos son los grids los cuales no sólo conectan redes de computadoras sino también usuarios y recursos para ofrecer y consumir servicios. Los pocos estándares hacen que los sistemas no sean interoperables entre sí, aislando los unos de otros. 
% Ajustar este conector
Ya que es difícil conectar sistemas de grids, entonces se habla del uso de portales de manera que se haga efectiva la coordinación y la interoperabilidad entre estos permitiendo además a distintas comunidades un trabajo colaborativo.\\

Se habla de un modelo de clasificación usado para identificar los "workflow-oriented Grid portals" basado en dos características: la habilidad de acceder a múltiples grids y el soporte para la solución cooperativa de problemas\cite{Kacsuk:2006}.\\

Uno de los campos que hace un uso intensivo de los sistemas grid es la bioinformática. Los biólogos poseen dificultad para instalar las aplicaciones de su área pero también saben que el uso de un único portal no sería suficiente para ofrecer todos los servicios que la comunidad necesita. Por eso se habla de la necesidad de transformar las aplicaciones web de solo HTML a algo más interactivo y que provea mecanismos para a los biólogos para realizar análisis de formas más accesibles y simples\cite{Gordon:2007}\cite{Javahery:2004}.\\

Es el usuario, de nuevo, quien debe hacer frente a la copia de datos a lo largo de diferentes servicios para ello crearon Mobyle\cite{Bertrand:2009}, un framework que permite a los administradores de portales, construir una red de servicios cooperativos que beneficiaran a la comunidad.\\

En el área de química, se usa GridChem\cite{GridChem}, “una organización virtual que provee el acceso a un sistema en grid para el uso de recursos en la computación de alto rendimiento” y Lattice QCD, un portal para compartir configuraciones generadas y basadas en las simulaciones de Lattice Data Grid Project\cite{QCD}. En física, se usa “Fusion Grid”\cite{Fusion} un software que asegura poder crear un ambiente de desarrollo cómodo para los usuarios nuevos mientras que para los administradores, facilita la exposición de los recursos para los stakeholders de la aplicación de forma segura, fiable y escalable.\\

En física, Cactus\cite{Cactus} es un framework que provee un ambiente colaborativo, modular y portable para la computación paralela de alto desempeño. Esta biblioteca permite desarrollar, en varios lenguajes, módulos que pueden ser ejecutados en grid.\\

En biología y ciencias médicas está BIRN\cite{BIRN}, un portal que ayuda a investigadores en ciencias biomédicas a procesar grandes volúmenes de datos independiente de su estructura, cómo serán accedidos o cómo se almacenarán, pero necesita de un permiso expreso de sus administradores por lo cual es necesario enviar una carta manifestando las intenciones.\\
% asdasd
% Aqui se debe resaltar el hecho de la diferencia entre portal científico (web portal) y 
Cabe resaltar que un portal científico, es una puerta de enlace que permite un acceso simple y eficiente a recursos computacionales hacia un sistema grid. Vale la pena mencionar que este acceso no implica necesariamente una interfaz web ni tampoco asegura la interoperabilidad con otros sistemas grid.\\

Un sistema grid es un conjunto de sistemas de cómputo (lease computadores) interconectados y usados para una misma finalidad, tal como el procesamiento de grandes volúmenes de datos\cite{Foster}. Y se caracterizan por :

\begin{itemize}
\item Coordinar recursos que no están sujetos a un control centralizado
\item Utiliza protocolos e interfaces abiertos
\item Ofrece cualidades no triviales de servicio
\end{itemize}

Una organización virtual (VO: virtual organization) son instituciones físicas o grupos lógicamente relacionados los cuales permite que recursos distribuidos geográficamente  puedan ser compartidos en el ambiente grid.
