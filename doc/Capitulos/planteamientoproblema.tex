\section{Planteamiento del Problema}


Las simulaciones por computador, realizadas a través de una aplicación científica, son una herramienta útil que permite a los investigadores de distintas áreas del conocimiento obtener datos de sus experimentos sin tener que reproducirlos en ambientes reales, permitiendo una reducción en los costos.\\

Son tambien denominadas experimentación en silicio y traen consigo numerosas ventajas como\cite{Tavera}:

\begin{itemize}
\item Mayor precisión y mejor calidad en los datos del experimento
\item Acceso a vastos conjuntos de datos
\item Incremento en la productividad
\end{itemize}

Hoy en día encontramos un volumen importante de simulaciones computacionales que necesitan ser procesadas adecuadamente para hacer un uso apropiado y eficiente de los recursos computacionales. La realidad es que estos recursos demandan de expertos no sólo para su gestión sino también para el uso de las herramientas de software que se ejecutan sobre ellos, algo que incrementa la complejidad del proyecto.\\

Conforme se especializan las aplicaciones para la computación científica, la curva de aprendizaje aumenta y con ello la dificultad con la que se instalan y administran dichas aplicaciones. Esto representa un desafío y un inconveniente aún mayor que el de resolver el problema inicialmente planteado para la simulación.\\

En conclusión, existen esfuerzos para la creación de soluciones (como portales científicos y workflows científicos) que permiten a los investigadores acceder al sistema sin barreras y de una forma más simple y rápida.

% https://taverna.incubator.apache.org/introduction/what-is-in-silico-experimentation.html
